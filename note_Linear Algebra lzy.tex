\documentclass{article}

\usepackage{amsmath, amsthm, amssymb}
\usepackage{graphicx}
\usepackage[UTF8]{ctex}
\usepackage{geometry}
\usepackage{bm}
\usepackage{extarrows}
\usepackage{enumitem}
\usepackage{float}
\usepackage{braket}
\usepackage{amsmath}
\usepackage{amssymb}
\usepackage{mathtools}
\usepackage{graphicx}
\usepackage{pifont}

\setcounter{MaxMatrixCols}{14}

\title{Essential Linear Algebra}
\author{路致远}

\newtheorem{definition}{定义}
\newtheorem{example}{例}
\newtheorem{theorem}{命题}
\newtheorem{lemma}{引理}
\newtheorem{question}{问题}
\newtheorem{answer}{解}[section]
\renewcommand{\theanswer}{}
\newtheorem{analysis}{分析}[section]
\renewcommand{\theanalysis}{}

\begin{document}
\maketitle
\section{线性空间与变换}
“我打开线代书一看,满篇都是变换。”——阮东
\subsection{线性空间}
\begin{definition}
    定义在集合$G$,域$F$上的线性空间$V$是指集合$G$及其上的加法和与$F$中元素的数乘运算。\\
    其中,$G$上加法构成Able群,数乘运算满足结合和分配律。\\
\end{definition}
\subsection{线性空间的结构}
\subsubsection{空间内结构:直和}
我们可以对线性空间进行直和分解。\\
\begin{definition}
    线性子空间$W_i$构成线性空间$V$的直和分解
    $$V=\bigoplus_{i=1}^{n} W_i$$
    假若
    $$\forall v \in V, \exists w_i \in W_i, \text{s.t.} v=\sum \limits_{i=1}^{n} w_i$$\\
    且这样的组合唯一。
\end{definition}
\begin{theorem}
    若线性空间$W=\sum \limits_{i=1}^{m} W_i$,则以下命题等价\\
    1.$W=\bigoplus \limits_{i=1}^{m} W_i$\\
    2.$\text{零元}0 \in W \text{表示为} W_i \text{元素和的方法唯一}$\\
    3.$W_i \cap \sum \limits_{j \neq i} W_j =0$\\
    4.$dim[W]= \sum \limits_{i=1}^{m} dim[W_i]$\\
\end{theorem}    
\begin{proof}
    \ \\
    2$\Rightarrow$1:\\
    若$a \in W,a=\sum \limits_{i=1}^{m} a_i=\sum \limits_{i=1}^{m} b_i;\ a_i, b_i \in W_i$\\
    则$0=\sum \limits_{i=1}^{m} (a_i - b_i) $\\
    从而$ a_i - b_i=0$即$a_i=b_i$\\
    3$\Rightarrow$2:\\
    若$0= \sum \limits_{i=1}^{m}a_i$\\
    即$-a_i= \sum \limits_{j \neq i} a_j$\\
    $\because -a_i \in W_i,\sum \limits_{j \neq i} a_j \in \sum \limits_{j \neq i} W_j$且$W_i \cap \sum \limits_{j \neq i} W_j =0$\\
    $\therefore -a_i,(\sum \limits_{j \neq i} a_j)=0$对所有i成立\\
    $\therefore\text{零元}0 \in W \text{表示为} W_i \text{元素和的方法唯一,即}0=0+0+\cdots +0$\\
    4$\Rightarrow$3:\\
    $dim[W_i \cap \sum \limits_{j \neq i} W_j]=dim[W_i]+dim[\sum \limits_{j \neq i} W_j]-dim[\sum \limits_{i=1}^{m} W_j]=0$\\
    从而$W_i \cap \sum \limits_{j \neq i} W_j$中只有零元。\\
    1$\Rightarrow$4:\\
    若$dim[w] \neq \sum \limits_{i=1}^{m} dim[W_i]$,即$\exists i \ \  \text{s.t.} \  W_i \cap \sum \limits_{j \neq i} W_j \neq \{0\}$\\
    对$0 \neq w \in W_i \cap \sum \limits_{j \neq i} W_j,0=w+(-w)=0+0$,与直和之定义矛盾。
\end{proof}
\begin{example}
    $\{\hat{e_i}\}_{i=1}^{n}$是空间$V$的一组基(最大无关组),则空间可以直和分解为
    $$V=\bigoplus_{i=1}^{n} \  span\{\hat{e_i}\}$$
\end{example}
\begin{example}
    对于有限维线性空间$V$上的变换$A$,$V$可以依据$A$的根子空间进行直和分解。
\end{example}
这一例子的证明见后。
\subsubsection{空间间结构:线性映射}
\begin{definition}
    映射$\phi[\ ]$是线性映射$ \{V,F\} \rightarrow \{W,F\} $,假若
    $$\phi[\lambda v +\mu u]=\lambda \phi[v]+\mu \phi[u];\  u,v \in V;\lambda,\mu \in F$$
    即所谓保线性。
\end{definition}
\subsection{线性映射}
为了更深入的了解线性映射$\phi[\ ]$,我们可以对映射的映像空间$W$和原像空间$V$进行直和分解,即引入一组基${\epsilon_i}_{i=1}^{n} \text{和} {e_{j}}_{j=1}^{m}$\\
设$\phi[e_j]=\sum \limits_{i=1}^{n} a_{ij} \epsilon_i$\\
我们把$a_{ij}$排成一纵列,
$$\begin{bmatrix}
    a_{1j}\\
    a_{2j}\\
    \cdots\\
    a_{nj}
\end{bmatrix}$$
并称之为其在映像空间$W$中的矩阵表示。应当注意的是,表示并不等于向量本身,同一向量允许不同表示。\\
把原像空间$V$的基横着排列起来,有
$$\phi[\begin{bmatrix}  e_1 & e_2 & \cdots & e_m   \end{bmatrix}]=
\begin{bmatrix}
    a_{11} & \null & a_{m1} \\
    a_{21} & \null & a_{m2} \\
    \cdots & \cdots& \cdots \\
    a_{n1} & \null & a_{mn} \\
\end{bmatrix}$$
我们就得到了线性映射的矩阵表示。
进一步地,对$\forall x \in V, x=\sum \limits_{j=1}^{m} x_j e_j $\\
由于变换保线性,映像之表示应是以上矩阵列向量之线性组合。由矩阵乘法定义,知道\textbf{矩阵右乘列向量,即将矩阵之列向量按右乘之列向量线性组合}(对左乘行向量亦是如此,这一点很重要,十分有助于思考)。\\
从而,映像之表示即\\
$$\phi[\begin{bmatrix}
    x_{j}\\
    x_{j}\\
    \cdots\\
    x_{j}
\end{bmatrix}]=
\begin{bmatrix}
    a_{11} & \null & a_{m1} \\
    a_{21} & \null & a_{m2} \\
    \cdots & \cdots& \cdots \\
    a_{n1} & \null & a_{mn} \\
\end{bmatrix}
\begin{bmatrix}
    x_{j}\\
    x_{j}\\
    \cdots\\
    x_{j}
\end{bmatrix}$$
特别的,我们喜欢这样的线性映射$\psi:\{V,F\} \rightarrow \{V,F\}$,称之为线性变换\\
其映射到自身,在$V$的基下表示为一个方阵,此时,我们可以对空间进行更精细的分解。
%\subsection{例题选讲}
%为了显示这种线性组合思想的用途,有必要做一些题。
\section{线性变换}
    “国庆之后,好日子就到头了。”——李思
\subsection{同构与同态}
\subsubsection{同构}
类比中学所学过的群论知识,可以定义同构:
\begin{definition}
    $线性映射\phi[\ ]:\{V_1,F\}\rightarrow\{V_2,F\}\text{是同构映射,假若}\phi \text{是双射,此时称}V_1,V_2\text{同构}$\\
    特别地,若$V_1=V_2$,称为自同构。
\end{definition}
\begin{theorem}
    同构映射以可逆矩阵表示。
\end{theorem}
\begin{proof}
\begin{lemma}
    同构映射保持线性独立。        
\end{lemma}
\begin{proof}
    用反证法。设$\{a_i\}_{i=1}^{m} \in V$是一组独立的元素,$\phi[\ ]$是同构映射。\\
    假若$\exists \{\lambda_i\}_{i=1}^{m} \in F$ 不全为零,s.t.
    $$\sum \limits_{i=1}^{m} \lambda_i \phi[a_i]=0$$\\
    由于保加法,$\phi[\sum \limits_{i=1}^{m} \lambda_i a_i]=0$\\
    而显然$\phi[0]=0$是唯一的。(保持加法唯一零元不变)\\
    $$\therefore \sum \limits_{i=1}^{m} \lambda_i a_i=0$$矛盾!\\
\end{proof}
从而同构将一最大无关组映射到另一最大无关组。用线性独立的列向量表示两组独立元素,设空间的维数为n,有
$$B=\begin{bmatrix}
    \beta_1 & \beta_2 & \cdots & \beta_n
\end{bmatrix}=T\begin{bmatrix}
    \alpha_1 & \alpha_2 & \cdots & \alpha_n
\end{bmatrix}=TA$$\\
其中,T是矩阵表示。由秩不等式(越乘越小)和AB满秩\\
$$n \geq \text{rank}[A]\text{(矩阵维数限制)},n=\text{rank}[B]=\text{rank}[TA] \leq \text{rank}[T]$$\\
$$\therefore \text{rank}[T]=n$$从而可逆。
\end{proof}
从而,我们得到有唯一解的线性方程组理论,寻找已知同构映射的原像的理论。\\
同构,顾名思义,具有完全相同的结构,是一个很高的要求。我们希望寻找弱一些的联系,并给出一般方程组的理论。同样类比中学群论知识,我们定义同态。
\subsubsection{同态}
\begin{definition}
    $\text{线性映射} \phi[\ ]:\{V_1,F\}\rightarrow\{V_2,F\}\text{是同态映射,假若}\phi \text{是满射,此时称}V_1,V_2\text{同态}$\\
\end{definition}
一个直观的图(请自动忽略图名)\\
\includegraphics[width=0.5\textwidth]{1.4.jpg}\\
完全地类比群论,定义同态核\\
\begin{definition}
    $V$之子集$ker[\phi]$是同构映射$T$之同态核,假若$T[x]=0,\forall x \in ker[\phi]$\\
\end{definition}
不难看出,同态核其实就是齐次方程的通解集合。类似陪集结构,同态核加上一个非核内元素,构成非齐次方程的一个解集。\\
以上,我们简要地介绍了思想,下面我们试图以之理解线性代数,使之更富逻辑,更加简明,以延续我们的好日子。
\subsubsection{习题选讲}
\begin{question}
    (秩不等式)试证:
    $$\text{rank}[AB] \geq \text{rank}[A]+\text{rank}[B]-n$$\\
    $$\text{rank}[ABC] \geq \text{rank}[AB]+\text{rank}[BC]-\text{rank}[B]$$\\
    其中,$A$是$m \times n$矩阵,$B$是$n \times p$矩阵,$C$是$p \times q$矩阵
\end{question}
\begin{proof}
\begin{lemma}
(维数定理) $\text{若线性空间}V\text{上有同态映射}\phi[\ ]:\{V,F\} \rightarrow \{Im[\phi],F\} \text{且其同态核为}ker[\phi],\text{映像空间为}Im[\phi],\\ \text{则} dim[V]=dim[ker[\phi]]+dim[Im[\phi]];$
\end{lemma}
\begin{proof}
    记$dim[ker[\phi]]=m,dim[Im[\phi]]=l,dim[V]=n$\\
    取同态核基$\{a_i\}_{i=1}^{m}$,将其补充为V的基$\{a_i\}_{i=1}^{n}$\\
    从而$\forall x \in V,x=\sum \limits_{i=1}^{n} \lambda_i a_i,\ \phi[x]=\phi[\sum \limits_{i=m+1}^{n} \lambda_i a_i]=\sum \limits_{i=m+1}^{n} \lambda_i \phi[a_i]$\\
    假若$\exists \{\mu_i\}_{i=m+1}^{n}$不全为零s.t.$\sum \limits_{i=m+1}^{n} \mu_i \phi[a_i]=0$\\
    即$\phi [\sum \limits_{i=m+1}^{n} \mu_i a_i] = 0$从而$\sum \limits_{i=m+1}^{n} \mu_i a_i \in ker[\phi]$\\
    这与基的定义矛盾!\\
    $\therefore \{\phi[a_i]\}_{i=m+1}^{n}$线性独立。由$\forall x \in V ,\phi[x]=\sum \limits_{i=m+1}^{n} \lambda_i \phi[a_i]$知道$\{\phi[a_i]\}_{i=m+1}^{n}$是映像空间的一组基。\\
    即$l=n-m$,即$dim[V]=dim[ker[\phi]]+dim[Im[\phi]]$\\
\end{proof}
先证第一个不等式,考虑$AB$是A分别作用于B之列向量上所构成同态映射$\phi_{\{B\} \rightarrow \{AB\}}$的映像空间\\
由维数定理:\\
$$dim[Im[A[B]]]=\text{rank}[AB]=dim[B\text{之列向量组}]-dim[ker[\phi_{\{B\} \rightarrow \{AB\}}]] =\text{rank}[B]-dim[ker[\phi_{\{B\} \rightarrow \{AB\}}]]$$\\
而由线性方程组理论$dim[ker[\phi_{\{B\} \rightarrow \{AB\}}]] \leq min\{m,n\}-\text{rank}[A]$\\
代入得$\text{rank}[AB] \geq \text{rank}[B]+\text{rank}[A]-min\{m,n\} \geq \text{rank}[B]+\text{rank}[A]-n$即证一式\\
再证二式:\\
同理,由维数定理有
$$\text{rank}[ABC]=\text{rank}[BC]-dim[ker[\phi_{\{BC\}\rightarrow \{ABC\}}]]$$\\
以及$$\text{rank}[AB]=\text{rank}[B]-dim[ker[\phi_{B \rightarrow AB}]]$$\\
从另一角度看,对B左乘C,实则是对B中的列向量线性组合,组合得到的BC中的列向量所张成的空间一定小于B中列向量所张成的。\\
从而$$dim[ker[\phi_{B \rightarrow AB}]] \geq dim[ker[\phi_{\{BC\}\rightarrow \{ABC\}}]]$$\\
$\therefore \text{rank}[ABC] \geq \text{rank}[BC]-dim[ker[\phi_{B \rightarrow AB}]]=\text{rank}[BC]+\text{rank}[AB]-\text{rank}[B]$\\
\end{proof}
\subsection{根子空间}
    “我们所要做的其实是一种简单的量子场论。”——鲜于中之
\subsubsection{准对角化}
    我们对于变换(方阵)的对角化是十分熟悉的,此时找到了一个特殊的对变换封闭的一维空间,其上的元素做变换,映像只是原像的简单数乘。
    然而,这样好的日子并不持续,我们知道存在不可对角化的矩阵。由此自然地想到,尽管一维不变空间不存在,可否有更高维数的不变空间?尽管简单数乘不存在,可否有简单矩阵的左乘替代之?\\
    不难看出,如果存在这样的高维不变子空间,若在各空间之基所\textbf{直和}成的基下,方阵可以“准对角化”,或者说分块对角化。\\
    (如果看不出,可以这样看)\\
    对矩阵$A\text{所表示的变换,记各子空间为}\{W_i\}_{i=1}^{m},\text{其基为}\{e_i^{(s)}\}_{s=1}^{k_i},\text{从而有}$\\
    $Ax=\sum \limits_{i=1}^{m} \sum \limits_{s=1}^{k_i}  x_i^{(s)} A e_i^{(s)}=\sum \limits_{i=1}^{m} \sum \limits_{s=1}^{k_i}  x_i^{(s)} [\sum \limits_{l=1}^{k_i} a_{i}^{(sl)} e_i^{(l)}]=\sum \limits_{i=1}^{m} \sum \limits_{l=1}^{k_i}  e_i^{(l)} [\sum \limits_{s=1}^{k_i} x_i^{(s)} a_{i}^{(sl)} ]$\\
    写成这个基础下的矩阵形式:\\
    $$A
    \begin{bmatrix}
    x_1^{(1)} \\ \cdots \\ x_1^{(k_1)} \\ x_2^{(1)} \\ \cdots \\ x_2^{(k_2)} \\ \cdots \\     
    \end{bmatrix}=
    \begin{bmatrix}
        a_1^{(11)}    & \cdots & a_1^{(k_1 1)}   & \null         & \null  & \null           & \cdots \\
        \cdots        & \null  & \cdots          & \null         & \null  & \null           & \null  \\
        a_1^{(1 k_1)} & \cdots & a_1^{(k_1 k_1)} & \null         & \null  & \null           & \null  \\
        \null         & \null  & \null           & a_1^{(11)}    & \cdots & a_1^{(k_1 1)}   & \null  \\
        \null         & \null  & \null           & \cdots        & \null  & \cdots          & \null  \\
        \null         & \null  & \null           & a_1^{(1 k_1)} & \cdots & a_1^{(k_1 k_1)} & \null  \\
        \cdots        & \null  & \null           & \null         & \null  & \null           & \null  \\
    \end{bmatrix}
    \begin{bmatrix}
        x_1^{(1)} \\ \cdots \\ x_1^{(k_1)} \\ x_2^{(1)} \\ \cdots \\ x_2^{(k_2)} \\ \cdots \\     
    \end{bmatrix}   
    $$\\
    现在你可以很明显地看出来了。\\
    进一步地,我们想问,对角块有应当具有怎样的形式?\\
\subsubsection{零化子与零化核}
    虽然我们已经知道答案是一个称为约当标准型的上三角形式(有些书里写的是下三角形式,但这是无关紧要的)。但是为了更自然地引出,我们需要耐心一点。\\
    首先来看我们熟知的知识,考虑方阵$A$的对角化:
    $$(A-\lambda_i I)\vec{\lambda_i}=0$$\\
    依据我们之前的理论,$\Lambda_i=span{\vec{\lambda_i}}$是矩阵$(A-\lambda_i I)$所代表同态变换的同态核。然而,此时同态的结构不重要,我们更愿意称之为零化核。对应的,多项式$\psi[A]=(A-\lambda_i I)$称为一个零化子\\
    (这个是我乱定义的,就不正式地写为定义了)\\
    例如,由凯莱-哈密顿定理,若多项式$\psi[X]=\text{det}[-A+XI]$,则$\psi[A]$就是一个以全空间为零化核的零化子。\\
    由凯莱-哈密顿定理,知道每一个方阵都存在全空间零化子。定义一个最小全空间零化子,其以全空间为零化核,且多项式形式的次数最低,记之为$m[A]$。\\
    因式分解有$$m[A]=\prod \limits_{i=1}^{n} (A-\mu_i I)^{t_i}$$\\
    对于可对角化的矩阵,知道其具有独立的本征矢量,从而,若想要以全空间为零化核,就得把所有本征方程中的$(A-\lambda_i I)$(所有相异的本征值)含进去,这样一来,就知道了
    \begin{theorem}
        $m[A]=\psi[A](\psi[X]=\text{det}(-A+XI))$,假若A可对角化,且具有两两不等的本征值。
    \end{theorem}
    对于不可对角化的矢量,由于所有本征矢量远非完备的基,但$\psi[A]$足以零化整个空间,这是我们所希望的。\\
    我们意识到,之所以不完备,是因为存在这样的矢量$x$,其虽不满足$(A-XI)x=0$,但是可以在更高次因子$(A-XI)^n$的情况况下零化$(A-XI)^nx=0$\\
    为了使基完备,我们引入如下矢量:\\
    $$\vec{s}_{ij}^{(k)} \in V \text{\ \ s.t.} (A-\lambda_iI)^{k}\vec{s}_{ij}^{(k)}=0 \text{且}(A-\lambda_iI)^{k-1}\vec{s}_{ij}^{(k)} \neq 0$$\\
    当$k=1$时,即为本征矢量。\\
    (对有限维的A,这样的$\vec{s}$总是存在的。这不难证明,只需证至少存在一个本征矢量即可,而方程$f[\lambda]=\text{det}[A-\lambda I]=0$是一个有限次多项式方程,其在$\mathbb{C}$上总是有解的。)\\
    下面我们来构造这样的完备基。\\
\subsubsection{约当标准型}
一个自然的想法是以阶数为标准进行直和分解。我们将会看到,这是不行的。\\
首先,我们取“零粒子空间”,即由$\vec{s}_{ij}^{(1)}$构成的本征矢量空间。
\begin{theorem}
    非简并(相异)本征值的本征矢独立。
\end{theorem}
\begin{proof}
    假设$\vec{s}_{11}^{(1)}$和$\vec{s}_{21}^{(1)}$分别是本征值$\lambda_1 \neq \lambda_2$的本征矢量。\\
    设$\exists \mu_1,\mu_2$不全为零s.t.\ $\mu_1 \vec{s}_{11}^{(1)}+\mu_2 \vec{s}_{21}^{(1)}=0$\\
    $$A[\mu_1 \vec{s}_{11}^{(1)}+\mu_2 \vec{s}_{21}^{(1)}]=\mu_1 \lambda_1 \vec{s}_{11}^{(1)} + \mu_2 \lambda_2 \vec{s}_{21}^{(1)}=\mu_1 (\lambda_1-\lambda_2) \vec{s}_{11}^{(1)}=0$$\\
    考虑$\vec{s}_{11}^{(1)} \neq 0,\lambda_1-\lambda_2 \neq 0$\\
    $$\therefore \mu_1=0=\mu_2$$\ 矛盾!
\end{proof}
对于简并的本征值,也可以有独立的本征矢,这样的一切矢量$\{\vec{s}_{ij}^{(1)}\}$构成了“零粒子空间”$W^{(1)}$的基,其个数比全空间的基数小。\\
我们考虑“一粒子空间”$W^{(2)}=span[\vec{s}_{ij}^{(2)}]$\\
其中$(A-\lambda_i)^2\vec{s}_{ij}^{(2)}=0$,不难注意$(A-\lambda_i)[(A-\lambda_i)\vec{s}_{ij}^{(2)}]=0$,即$(A-\lambda_i)\vec{s}_{ij}^{(2)} \in W_1 \  \star$\\
我们将其中的独立矢量挑出来,组成一粒子空间$W^{(2)}$的一组基$\{\vec{s}_{ij}^{(2)}\}$。\\
$\cdots$以此类推,得到各个空间。\\
我们想要证明以下的命题
\begin{theorem}
    $$W^{(m)}+W^{(n)}=W^{(m)} \oplus W^{(n)},for\ n \neq m$$
\end{theorem}
一个证明的尝试展示如下\\
\begin{proof}
    不妨设$m \geq n$,由式$\star,\ \forall \ \text{基}\  \vec{s}_{ik}^{(m)} \in W^{(m)},(A-\lambda_i I)^{(m-n)} \vec{s}_{ik}^{(m)} \in W^{(n)} $\\
    设$ \{\nu_{ij}\}$不全为零,$(A-\lambda_i I)^{(m-n)} \vec{s}_{ik}^{(m)}=\sum \limits_{j=1}^{l_i^{(n)}} \nu_{ij} \vec{s}_{ij}^{(n)}$\\
    只需证任一$W^{(m)}$基不能被$W^{(n)}$基线性表示,用反证法:\\
    设$\exists \{\mu_{ij}\}$不全为零,$\vec{s}_{ik}^{(m)}=\sum \limits_{j} \mu_{ij} \vec{s}_{ij}^{(n)}$\\
    代入得$(A-\lambda_i I)^{(m-n)} \sum \limits_{j=1}^{l_i^{(n)}} \mu_{ij} \vec{s}_{ij}^{(n)} = \sum \limits_{j=1}^{l_i^{(n)}} \nu_{ij} \vec{s}_{ij}^{(n)}$\\
    设所研究线性空间的维数为p,由列向量 $\{\vec{s}_{ij}^{(n)}\}_{j=1}^{l_i^{(n)}}$所排列成的是$p \times l_i^{(n)}$的矩阵$S$。由于各列向量独立,可以对其进行归一正交化,对最后的结果并无影响,不过是$\mu_i$和$\nu_i$重新组合了而已。
    假设$S$已经归一正交化好。有矩阵形式:\\
    $$S^{\dagger} (A-\lambda_i I)^{(m-n)} S
    \begin{bmatrix}
    \mu_{i1} \\ \mu_{i2} \\ \cdots \\ \mu_{i l_i^{(n)}}    
    \end{bmatrix}=
    \begin{bmatrix}
    \nu_{i1} \\ \nu_{i2} \\ \cdots \\ \nu_{i l_i^{(n)}}    
    \end{bmatrix}
    $$
    由矩阵不等式,记$(A-\lambda_i I)^{(m-n)}=C$\\
    需要证明$S^{\dagger} C S \vec{\mu}=\vec{\nu}$无解或没有唯一解,而$\text{rank}[C] < p,\text{rank}[S^{\dagger}]=\text{rank}[S]=l_i^{(n)}$\\
    我们可能想证$\text{rank}[S^{\dagger} C S] < l_{i}^{(n)}$这是一个很符合直觉的想法\\
    很遗憾,\textbf{但这是错的}。我们很容易就能举出一个反例:\\
    $$
    \begin{bmatrix}
        \frac{1}{\sqrt{2}} & \frac{1}{\sqrt{2}} & 0                  & 0                 \\
        0                  &                    & \frac{1}{\sqrt{2}} & \frac{1}{\sqrt{2}}\\
    \end{bmatrix}
    \begin{bmatrix}
        1 & \null & \null & \null \\
    \null & 1     & \null & \null \\
    \null & \null & 1     & \null \\
    \null & \null & \null & 0     \\
    \end{bmatrix}
    \begin{bmatrix}
        \frac{1}{\sqrt{2}} & 0                  \\
        \frac{1}{\sqrt{2}} & 0                  \\
        0                  & \frac{1}{\sqrt{2}} \\
        0                  & \frac{1}{\sqrt{2}} \\
    
    \end{bmatrix}
    =
    \begin{bmatrix}
        1 & 0           \\
        0 & \frac{1}{2} \\
    \end{bmatrix}
    $$\\
\end{proof}
这种分解方法\textbf{远非直和,命题5是大错特错},而且其不变性质也极差,考虑:\\
$$A\vec{s}_{ij}^{(m)}=(A-\lambda_i I)\vec{s}_{ij}^{(m)}+\lambda_i \vec{s}_{ij}^{(m)}$$\\
由式$\star$,知道$(A-\lambda_i I)\vec{s}_{ij}^{(m)} \in W^{(m-1)}$\\
$\therefore A\vec{s}_{ij}^{(m)}=\lambda_i \vec{s}_{ij}^{(m)}+\sum \limits_{j=1}^{l_{i}^{(m-1)}} p_{ij}^{m(m-1)} \vec{s}_{ij}^{(m-1)} \star \star$\\
这远非不变直和分解,更无法将矩阵准对角化。不难想象,在这组“基”下的表示矩阵,会出现$[p_{ij}^{m(m-1)}]$这样的非对角分块。\\
但这并非毫无价值,式$\star \star$启示我们,“m粒子空间”与“m-1粒子空间”“联合”不变。假若我们给每一个$\vec{s}_{ij}^{(m)}$都指定一组“直系亲属”$\{(A-\lambda_i I)^n \vec{s}_{ij}^{(m)}=\vec{s}_{ij}^{(m-n)}\}_{n=0}^{m-1}$\\
这样的矢量构成一个不变子空间$J_{ij}$,且其满足:
$$A \vec{s}_{ij}^{(t)}=(A-\lambda_i I) \vec{s}_{ij}^{(t)}+\lambda_i A \vec{s}_{ij}^{(t)}=\vec{s}_{ij}^{(t-1)}+\lambda_i \vec{s}_{ij}^{(t)}$$
如果写成矩阵形式,即:
$$
A \begin{bmatrix} \vec{s}_{ij}^{(1)} & \vec{s}_{ij}^{(2)} & \cdots & \vec{s}_{ij}^{(m)} \end{bmatrix}=
\begin{bmatrix} \vec{s}_{ij}^{(1)} & \vec{s}_{ij}^{(2)} & \cdots & \vec{s}_{ij}^{(m)} \end{bmatrix}
\begin{bmatrix}
    \lambda_i & 1         & 0     & \cdots & 0         \\
    0         & \lambda_i & 1     & \cdots & 0         \\
    \null     & \null     & \null & \cdots & \null     \\
    0         & 0         & 0     & \cdots & \lambda_i \\
\end{bmatrix}
$$\\
从而自然地完成了循环分解、得出了约当标准型。\\
而考虑以全空间为零化核的零化子$\psi[A]=\text{det}[-A+XI]|_{X=A}=\prod \limits_{i=1}^{n} (A-\lambda_i I)^{t_i}$\\
对其中的每一个因子$(A-\mu_i I)^{t_i}$,可以首先找到$\vec{s}_{ij}^{t_i}$,再依次左乘,得到整个$J_{ij}$,并在这个基下,给出准对角化的$A$的一个对角块。\\
\subsubsection{习题选讲:计算技巧}
“证明结束后,应当撤走所有的梯子。”——高斯\\
\ \\
首先做一个简单的计算,以熟悉我们刚刚所讲的东西。
\begin{question}
    试将: $$A=
    \begin{bmatrix}
        3 & 1 & -1 \\
        2 & 2 & -1 \\
        2 & 2 & 0  \\
    \end{bmatrix}$$
    相似变换为上三角阵。
\end{question}
\begin{answer}
    本征方程$$\text{det}[-A+\lambda I]=(\lambda-2)^2(\lambda-1)$$\\
    零化子$$\psi[A]=(A-2I)^2(A-1)$$\\
    对$(A-I)$,本征矢量$\vec{s}_{11}^{(1)}=\begin{bmatrix} 1 \\ 0 \\ 2 \end{bmatrix}$\\
    对$(A-2I)^2=
    \begin{bmatrix}
        1 & -1 & 0 \\
        0 & 0  & 0 \\
        2 & -2 & 0 \\
    \end{bmatrix},\vec{s}_{21}^{(2)}=\begin{bmatrix} 1 \\ 1 \\ 0 \end{bmatrix},\vec{s}_{21}^{(1)}=\begin{bmatrix} 2 \\ 2 \\ 4 \end{bmatrix}$\\
    $$\therefore A \begin{bmatrix} \vec{s}_{11}^{(1)} \vec{s}_{21}^{(1)} \vec{s}_{21}^{(2)} \end{bmatrix}=\begin{bmatrix} \vec{s}_{11}^{(1)} \vec{s}_{21}^{(1)} \vec{s}_{21}^{(2)} \end{bmatrix}
    \begin{bmatrix}
        1 & \null & \null \\
    \null & 2     & 1     \\
    \null & \null & 2     \\
    \end{bmatrix}$$
    即$$A=\frac{1}{4}
    \begin{bmatrix}
        1 & 2 & 1 \\
        0 & 2 & 1 \\
        2 & 4 & 0 
    \end{bmatrix}
    \begin{bmatrix}
        1 & \null & \null \\
    \null & 2     & 1     \\
    \null & \null & 2     \\
    \end{bmatrix}
    \begin{bmatrix}
         4 & -4 & 0 \\
        -2 &  2 & 1 \\
         4 &  0 &-2  
    \end{bmatrix}
    $$\\
    即为所求。
\end{answer}
然后是一个作业题,虽然比较水,但是一个明确思路的好例子
\begin{question}
    试证:约当标准型$J$与之转置相似。
\end{question}
这个例子可以用于证明任一方阵与之转置相似。同时,这也导致部分参考书的约当块是下三角形式。
\begin{analysis}
    约当标准型的意义用下式(列向量的线性组合)说明:
    $$[s_{ij}^{(n)}]J_{ij}=[\lambda_i s_{ij}^{(n)}+s_{ij}^{(n-1)}]$$
    这是从左到右按照n从小到大排列的。如果将n从大到小排列,而保持以上线性组合关系,我们就会发现,$J_{ij}$应当转置。\\
    改变排列顺序的矩阵即是反对角矩阵。从而我们写下证明。
\end{analysis}
\begin{proof}
    注意到
    $$K=
    \begin{bmatrix}
    0         & 0         & \cdots & 0      & 1         \\
    0         & 0         & \cdots & 1      & 0         \\
    \null     & \null     & \cdots & \null  & \null     \\
    1         & 0         & \cdots & 0      & 0         \\
    \end{bmatrix}s.t.K^2=I,\text{即}K=K^{-1}
    $$
    且其满足
    $$KJK^{-1}=J^{T}$$
    从而相似。
\end{proof}
以上只是机械地计算。下面我们看两个巧算的例子;
\begin{question}
    求矩阵
    $$A=
    \begin{bmatrix}
        2 & 0 & 0 & 0 & 0 & 0 \\
        1 & 2 & 0 & 0 & 0 & 0 \\
       -1 & 0 & 2 & 0 & 0 & 0 \\
        0 & 1 & 0 & 2 & 0 & 0 \\
        1 & 1 & 1 & 1 & 2 & 0 \\
        0 & 0 & 0 & 0 & 1 &-1 \\
    \end{bmatrix}
    $$
    之约当标准形。
\end{question}
\begin{analysis}
    本题看似计算量较大,但注意到并不要求求出变换矩阵,于是我们通过观察,可以直接地写出原矩阵的约当标准型。\\
    易知特征多项式$\phi[\lambda]=[\lambda-(-1)][\lambda-2]^5$\\
    从而A有两个本征值。对应-1本征值的本征矢量只有一个,给出约当标准型
    $$J_{(-1)}=
    \begin{bmatrix}
        1
    \end{bmatrix}
    $$
    而对于本征值2,考虑$B=A-2I=
    \begin{bmatrix}
        0 & 0 & 0 & 0 & 0 & 0 \\
        1 & 0 & 0 & 0 & 0 & 0 \\
       -1 & 0 & 0 & 0 & 0 & 0 \\
        0 & 1 & 0 & 0 & 0 & 0 \\
        1 & 1 & 1 & 1 & 0 & 0 \\
        0 & 0 & 0 & 0 & 1 &-3 \\
    \end{bmatrix}$\\
    观察行向量的线性相关性可以知道$\text{rank}[B]=4$,从而该方程有2个独立的本征矢量,我们得到“零粒子空间”$dim[W^{(1)}]=2$\\
    根子空间还剩下3个基向量待决定。\\
    对于“一粒子空间”$W^{(2)}$,其至多有两个基向量,因为其必须要能“生成”一粒子空间。$B^2$是不得不算的。\\
    $$B^2=
    \begin{bmatrix}
        0 & 0 & 0 & 0 & 0 & 0 \\
        0 & 0 & 0 & 0 & 0 & 0 \\
        0 & 0 & 0 & 0 & 0 & 0 \\
        1 & 0 & 0 & 0 & 0 & 0 \\
        0 & 1 & 0 & 0 & 0 & 0 \\
        1 & 1 & 1 & 1 &-3 & 9 \\
    \end{bmatrix}
    $$
    同样,$\text{rank}[B^2]=3$,有三个独立的解,而其中两个是零粒子空间的,一粒子空间只有一个,即$dim[W^{(2)}]=1$\\
    同样的理由,剩下两个矢量分别包含于二粒子空间和三粒子空间,$dim[W^{(3)}]=dim[W^{(4)}]=1$\\
    于是,我们清楚,有两个小约当块,一个对应一阶的循环空间,一个对应四阶的循环空间。约当块即:
    $$J_{(2),1}=
    \begin{bmatrix}
        2
    \end{bmatrix}
    $$
    $$J_{(2),2}=
    \begin{bmatrix}
        2 & 1     & \null & \null \\
    \null & 2     & 1     & \null \\
    \null & \null & 2     & 1     \\
    \null & \null & \null & 2     \\
    \end{bmatrix}
    $$
    而所求的约当标准型即$diag\{J_{(-1)},J_{(2),1},J_{(2),2}\}$\\
    而显然,以上的过程只是我们的分析,“n粒子空间”不显于讲义的任意角落,明显不能书写。于是我们决定向高斯学习。\\
\end{analysis}

\begin{answer}
    经计算,
    $$J=
    \begin{bmatrix}
        1 & 0 & 0 & 0 & 0 & 0 \\
        0 & 2 & 0 & 0 & 0 & 0 \\
        0 & 0 & 2 & 1 & 0 & 0 \\
        0 & 0 & 0 & 2 & 1 & 0 \\
        0 & 0 & 0 & 0 & 2 & 1 \\
        0 & 0 & 0 & 0 & 0 & 2 \\
    \end{bmatrix}
    $$
    即为所求约当标准型。
\end{answer}
运用上面的经验,我们继续看这样一个直接计算更加困难的问题。\\
\begin{question}
    若14阶矩阵
    $$N=
    \begin{bmatrix}
        0     & 1     & \null  & \null  \\
        \null & 0     & 1      & \null  \\
        \null & \null & \cdots & \null  \\
        \null & \null & \null  & 0      \\
    \end{bmatrix}
    $$
    求$N^4$的约当标准型及根子空间。
\end{question}
\begin{analysis}
    N的本征值自然全是零,而N的乘方性质是我们所熟悉的,即乘方一次,对角线平移一格。$N^4$则平移四格,以第一行为例,前四列都是0,第五列为1。\\
    容易知道,
    $S^{(1)}_{11}=
    \begin{bmatrix}
        1 & 0 & 0 & 0 & \cdots & 0 
    \end{bmatrix}^T$,
    $S^{(1)}_{21}=
    \begin{bmatrix}
        0 & 1 & 0 & 0 & \cdots & 0 
    \end{bmatrix}^T$,
    $S^{(1)}_{31}=
    \begin{bmatrix}
        0 & 0 & 1 & 0 & \cdots & 0 
    \end{bmatrix}^T$,
    $S^{(1)}_{41}=
    \begin{bmatrix}
        0 & 0 & 0 & 1 & \cdots & 0 
    \end{bmatrix}^T$是本征矢量,构成零粒子空间。\\
    考察一粒子空间,有$(N^4)^2=N^8$,前八列都是0,第九列为1。排除掉零粒子空间,有四个独立向量,分别在第5、6、7、8行为1,分别记之$S^{(2)}_{11},S^{(2)}_{21},S^{(2)}_{31},S^{(2)}_{41}$。\\
    不难看出,$(N^4)S^{(2)}_{k1}=S^{(1)}_{k1},\ k=1,2,3,4$\\
    同理,我们可以写出$S^{(3)}_{11},S^{(3)}_{21},S^{(3)}_{31},S^{(3)}_{41}$仍然满足上述生成关系,但对于三粒子空间,我们只剩两个独立的向量($14=3 \times 4 \cdots 2$)\\
    其应该出现在$S^{(4)}_{11},S^{(4)}_{21}$这两个最靠前的循环子空间,其分别在第13行和第14行为1。\\
    于是我们得到了答案。
\end{analysis}
\begin{answer}
    注意到:\\
    $N^4=
    \begin{bmatrix}
        1 & 0 & 0 & 0 & 0 & 0 & 0 & 0 & 0 & 0 & 0 & 0 & 0 & 0 \\
        0 & 0 & 0 & 0 & 1 & 0 & 0 & 0 & 0 & 0 & 0 & 0 & 0 & 0 \\
        0 & 0 & 0 & 0 & 0 & 0 & 0 & 0 & 1 & 0 & 0 & 0 & 0 & 0 \\
        0 & 0 & 0 & 0 & 0 & 0 & 0 & 0 & 0 & 0 & 0 & 1 & 0 & 0 \\
        0 & 1 & 0 & 0 & 0 & 0 & 0 & 0 & 0 & 0 & 0 & 0 & 0 & 0 \\
        0 & 0 & 0 & 0 & 0 & 1 & 0 & 0 & 0 & 0 & 0 & 0 & 0 & 0 \\
        0 & 0 & 0 & 0 & 0 & 0 & 0 & 0 & 0 & 1 & 0 & 0 & 0 & 0 \\
        0 & 0 & 0 & 0 & 0 & 0 & 0 & 0 & 0 & 0 & 0 & 0 & 1 & 0 \\
        0 & 0 & 1 & 0 & 0 & 0 & 0 & 0 & 0 & 0 & 0 & 0 & 0 & 0 \\
        0 & 0 & 0 & 0 & 0 & 0 & 1 & 0 & 0 & 0 & 0 & 0 & 0 & 0 \\
        0 & 0 & 0 & 0 & 0 & 0 & 0 & 0 & 0 & 0 & 1 & 0 & 0 & 0 \\
        0 & 0 & 0 & 0 & 0 & 0 & 0 & 0 & 0 & 0 & 0 & 0 & 0 & 1 \\
        0 & 0 & 0 & 1 & 0 & 0 & 0 & 0 & 0 & 0 & 0 & 0 & 0 & 0 \\
        0 & 0 & 0 & 0 & 0 & 0 & 0 & 1 & 0 & 0 & 0 & 0 & 0 & 0 \\
    \end{bmatrix}\\   
    \begin{bmatrix}
        0 & 1 & 0 & 0 & 0 & 0 & 0 & 0 & 0 & 0 & 0 & 0 & 0 & 0 \\
        0 & 0 & 1 & 0 & 0 & 0 & 0 & 0 & 0 & 0 & 0 & 0 & 0 & 0 \\
        0 & 0 & 0 & 1 & 0 & 0 & 0 & 0 & 0 & 0 & 0 & 0 & 0 & 0 \\
        0 & 0 & 0 & 0 & 0 & 0 & 0 & 0 & 0 & 0 & 0 & 0 & 0 & 0 \\
        0 & 0 & 0 & 0 & 0 & 1 & 0 & 0 & 0 & 0 & 0 & 0 & 0 & 0 \\
        0 & 0 & 0 & 0 & 0 & 0 & 1 & 0 & 0 & 0 & 0 & 0 & 0 & 0 \\
        0 & 0 & 0 & 0 & 0 & 0 & 0 & 1 & 0 & 0 & 0 & 0 & 0 & 0 \\
        0 & 0 & 0 & 0 & 0 & 0 & 0 & 0 & 0 & 0 & 0 & 0 & 0 & 0 \\
        0 & 0 & 0 & 0 & 0 & 0 & 0 & 0 & 0 & 1 & 0 & 0 & 0 & 0 \\
        0 & 0 & 0 & 0 & 0 & 0 & 0 & 0 & 0 & 0 & 1 & 0 & 0 & 0 \\
        0 & 0 & 0 & 0 & 0 & 0 & 0 & 0 & 0 & 0 & 0 & 0 & 0 & 0 \\
        0 & 0 & 0 & 0 & 0 & 0 & 0 & 0 & 0 & 0 & 0 & 0 & 1 & 0\\
        0 & 0 & 0 & 0 & 0 & 0 & 0 & 0 & 0 & 0 & 0 & 0 & 0 & 1 \\
        0 & 0 & 0 & 0 & 0 & 0 & 0 & 0 & 0 & 0 & 0 & 0 & 0 & 0 \\
    \end{bmatrix}
    \begin{bmatrix}
        1 & 0 & 0 & 0 & 0 & 0 & 0 & 0 & 0 & 0 & 0 & 0 & 0 & 0 \\
        0 & 0 & 0 & 0 & 1 & 0 & 0 & 0 & 0 & 0 & 0 & 0 & 0 & 0 \\
        0 & 0 & 0 & 0 & 0 & 0 & 0 & 0 & 1 & 0 & 0 & 0 & 0 & 0 \\
        0 & 0 & 0 & 0 & 0 & 0 & 0 & 0 & 0 & 0 & 0 & 0 & 1 & 0 \\
        0 & 1 & 0 & 0 & 0 & 0 & 0 & 0 & 0 & 0 & 0 & 0 & 0 & 0 \\
        0 & 0 & 0 & 0 & 0 & 1 & 0 & 0 & 0 & 0 & 0 & 0 & 0 & 0 \\
        0 & 0 & 0 & 0 & 0 & 0 & 0 & 0 & 0 & 1 & 0 & 0 & 0 & 0 \\
        0 & 0 & 0 & 0 & 0 & 0 & 0 & 0 & 0 & 0 & 0 & 0 & 0 & 1 \\
        0 & 0 & 1 & 0 & 0 & 0 & 0 & 0 & 0 & 0 & 0 & 0 & 0 & 0 \\
        0 & 0 & 0 & 0 & 0 & 0 & 1 & 0 & 0 & 0 & 0 & 0 & 0 & 0 \\
        0 & 0 & 0 & 0 & 0 & 0 & 0 & 0 & 0 & 0 & 1 & 0 & 0 & 0 \\
        0 & 0 & 0 & 1 & 0 & 0 & 0 & 0 & 0 & 0 & 0 & 0 & 0 & 0\\
        0 & 0 & 0 & 0 & 0 & 0 & 0 & 1 & 0 & 0 & 0 & 0 & 0 & 0 \\
        0 & 0 & 0 & 0 & 0 & 0 & 0 & 0 & 0 & 0 & 0 & 1 & 0 & 0 \\
    \end{bmatrix}    
    $\\
    即为所求。
\end{answer}
\subsection{习题选讲:DN分解}
(这个名词是我瞎造的,不要引用)\\
由之前的作业,我们知道,对任意n阶方阵A,$A=D+N$,且$D$,$N$可交换,即一个变换是由其可对角化部分$D$和幂零部分$N$组成。
\begin{question}
    若$A$是n维复线性空间$V$上的一个线性变换,$D$是$A$的可对角化部分,试证,对多项式$g[A]$,其可对角化部分为$g[D]$\\
\end{question}
\begin{proof}
    设$g[A]=\sum \limits_{k=0}^{h} g_k A^k$,采用我们之前使用的基$\{s_{ij}^{(n)}\}$,在此基下,$A$表示为一个约当标准型
    有$A s_{ij}^{(n)}=\lambda_i s_{ij}^{(n)}+s_{ij}^{(n-1)}$\\
\begin{lemma}
    $A^k s_{ij}^{(n)}=\sum \limits_{m=0}^{k} C_k^m \lambda_i^m s_{ij}^{(n-k+m)}$\\
\end{lemma}
\begin{proof}
    用归纳法证明之:
    $A^1 s_{ij}^{(n)}=\lambda_i s_{ij}^{(n)}+s_{ij}^{(n-1)}$\\
    若$A^{(k-1)} s_{ij}^{(n)}=\sum \limits_{m=0}^{k-1} C_{k-1}^m \lambda_i^m s_{ij}^{(n-k+1+m)}$\\
    则$A A^{(k-1)} s_{ij}^{(n)}=\sum \limits_{m=0}^{k-1} C_{k-1}^m \lambda_i^{m+1} s_{ij}^{(n-k+1+m)}+\sum \limits_{m=0}^{k-1} C_{k-1}^m \lambda_i^m s_{ij}^{(n-k+m)}=\sum \limits_{m=1}^{k-1} (C_{k-1}^m+C_{k-1}^{m-1}) \lambda_i^m s_{ij}^{(n-k+m)}+s_{ij}^{n-k}+s_{ij}^{n}=\sum \limits_{m=0}^{k} C_{k}^m \lambda_i^{m} s_{ij}^{(n-k+m)}$\\
    从而证得断言。\\
\end{proof}
    将$A$分解为可对角化/幂零:
    $$A s_{ij}^{(n)}=(D+N) s_{ij}^{(n)}=\lambda_i s_{ij}^{(n)}+s_{ij}^{(n-1)}$$\\
    即
    $$D s_{ij}^{(n)}=\lambda_i s_{ij}^{(n)} $$\\
    同理,由引理
    $$A^k s_{ij}^{(n)}=\sum \limits_{m=0}^{k} C_k^m \lambda_i^m s_{ij}^{(n-k+m)}=\lambda_i^k s_{ij}^{(n)}+\sum \limits_{m=0}^{k-1} C_k^m \lambda_i^m s_{ij}^{(n-k+m)}$$\\
    $$\therefore \text{(g[A]的可对角部分)} D[g[A]] s_{ij}^{(n)}= \sum \limits_{k=0}^{h} g_k \lambda_i^k s_{ij}^{(n)}$$\\
    而$D s_{ij}^{(n)}=\lambda_i s_{ij}^{(n)},g[D] s_{ij}^{(n)}= \sum \limits_{k=0}^{h} g_k \lambda_i^k s_{ij}^{(n)}$\\
    $$\therefore D[g[A]]=g[D]$$\\    
\end{proof}
现在我们已经基本清楚如何使用这样的分解。下面做一个简单的练习:
\begin{question}
    若n阶实方阵A满足$A^2+I=0$,试证明:$A \sim \begin{bmatrix} 0_{\frac{n}{2} \times \frac{n}{2}} & -I_{\frac{n}{2} \times \frac{n}{2}} \\ I_{\frac{n}{2} \times \frac{n}{2}} & 0_{\frac{n}{2} \times \frac{n}{2}} \end{bmatrix}$;
\end{question}
\begin{proof}
    首先证n是偶数。$\text{det}[A^2]=[\text{det}[A]]^2=\text{det}[-I]=(-1)^n$,从而n为偶数。\\
    设A的约当标准型为:\\$\exists P $可逆,s.t. $PAP^{-1}=D+N$,其中,D是对角的,N是m+1阶幂零的,且$m+1 \leq n$\\
    考虑$D,N$可交换,$A^2=(D+N)^2=D^2+2DN+N^2=-I$,其中只有$D^2$对对角元有贡献,从而$D=diag[\pm i]$\\
    因为A是实矩阵,$\text{tr}[A]=\text{tr}[D] \in \mathbb{R} $,故D之对角元正负参半。不妨正值全部排列于左上,负值全部排列于右下。
    容易知道$D^{-1}=-D$\\
    $\because -AA=I \therefore A^{-1}=-A=-D-N$\\
    而注意到$(D+N)[(-D)(I+DN+D^2 N^2+ \cdots +D^m N^m)]=I$\\(事实上,这一步是形式化地读写$\frac{1}{D+N}$再做“级数展开”得到)\\
    $\therefore [(-D)(I+DN+D^2 N^2+ \cdots +D^m N^m)]=-D-N$\\
    即
    $$
    N+DN^2-N^3-DN^4+N^5+\cdots + (-D^{m+1})N^m=-N
    $$
    两侧同乘以$N^{m-1}$,有:
    $$N^m=-N^m$$ $\therefore N^m=0$\\
    带入上上式,有
    $$
    N+DN^2-N^3-DN^4+N^5+\cdots + (-D^{m})N^{m-1}=-N
    $$
    两侧同乘以$N^{m-2}$,有:
    $$N^{m-1}=-N^{m-1}$$ $\therefore N^{m-1}=0$\\
    依此类推,得到$N=0$\\
    $\therefore$A可对角化,且$A \sim diag[+i,+i,\cdots,+i,-i,-i,\cdots,-i]=\Lambda$\\
    而对于$B=\begin{bmatrix} 0_{\frac{n}{2} \times \frac{n}{2}} & -I_{\frac{n}{2} \times \frac{n}{2}} \\ I_{\frac{n}{2} \times \frac{n}{2}} & 0_{\frac{n}{2} \times \frac{n}{2}} \end{bmatrix}$
    带入本征方程$\text{det}[B-\lambda_i I]=0$(这个的快速算法就是直接对中心的二阶行列式展开计算)\\
    解得$\lambda_i= \pm i$,带入本征方程,可以求得n个独立的本征矢量,从而B可对角化。即
    $$B \sim \Lambda \sim A$$
\end{proof}
然后以此复习一下线性方程组理论,用一个较简单的水题看看它在坏日子里面有什么变化
\begin{question}
    $A$是有限维线性空间$V$上的一个变换,且$\text{rank}[A]=1$,试证:$A$要么可以对角化,要么是幂零的。
\end{question}
\begin{proof}
    设$dim[V]=n$,由线性方程组理论,$dim[ker[\phi_A]]=n-\text{rank}[A]=n-1$\\
    设核基为$\{s^i\}_{i=1}^{n-1}$,$s^n$为之余下的一个基\\
    $$\therefore A s^i =0,for 1 \leq i \leq n-1$$
    可设$$A s^n=\lambda s^n + \sum \limits_{i=1}^{n-1}\mu_i s^i$$\\
    1.若$\lambda \neq 0$\\
    则$$A [s^n + \sum \limits_{i=1}^{n-1} \frac{\mu_i}{\lambda} s^i]=\lambda [s^n + \sum \limits_{i=1}^{n-1} \frac{\mu_i}{\lambda} s^i]$$\\
    令$s^n + \sum \limits_{i=1}^{n-1} \frac{\mu_i}{\lambda} s^i$为一个新基,与核基一起构成完备的基,在该基下,$A$被对角化。\\
    2.若$\lambda = 0$\\
    即$A s^n=\sum \limits_{i=1}^{n-1}\mu_i s^i$,则$A^2 s^i=0 ,for 1 \leq i \leq n$\\
    即为幂零的。
\end{proof}
最来看一个无穷维的简单例子(以防止被期末偷袭)
\begin{question}
    若$V$是无限可微实函数所组成之线性空间,导子$\mathcal{D}$是$V$上一个线性变换。求之本征值、本征矢及根子空间。
\end{question}
\begin{answer}
    本征方程$(\mathcal{D}-\lambda)f(x)=0$\\
    解得$f(x)=\mathbb{R}e^{\lambda x}$,即为所求对应本征值$\lambda$的本征矢量所在空间。\\
    根子方程$(\mathcal{D}-\lambda)^n f(x)=0$\\
    由微分方程理论,根子空间为$e^{\lambda x}\mathbb{R}[x]$,$\mathbb{R}[x]$是x的实系数多项式。\\
    即为所求。
\end{answer}

\section{内积空间}
“长度是什么,我们还没有定义好。”——艾颖华
\subsection{幺正变换与正定性}
    笔者并没有太多超出李思老师讲义的内容,故不作详述,只将重点罗列如下,并讨论一些解题技巧。\\
    (1)幺正变换的定义与性质,厄密矩阵的定义与性质,正定的定义与性质(正定的定义基于实对称,这一点值得注意);\\
    (2)SVD,QR分解,极分解,广义逆与MP逆。\\
    (3)最小二乘法、格拉姆矩阵。\\
\subsection{习题选讲:对称构造}
首先看一个作业题;
\begin{question}
    试证明两正定矩阵$A,B$之积$AB$仍正定。
\end{question}
\begin{analysis}
    本题看似简单,实则并不显然。主要的原因就是$AB$缺乏对称性,使得许多工作难以开展。同时知道,相似变换不改变正定性,我们希望构造一个对称性较好的、同时与$AB$相似的矩阵。
    首先想到的是$AB+BA$,但对于多项式,其很难与一个单项式相似。从而,我们转向$ABA$,但是由于矩阵次数的冲突(一个是“2次”,一个是“3次”),并不相似。于是我们对称地调整次数,终于得到$A^{\frac{1}{2}}BA^{\frac{1}{2}}$这一符合要求的矩阵。
\end{analysis}
\begin{proof}
    由于A是实对称正定矩阵\\
    $\therefore \exists P$正交$s.t.A=P diag(\lambda_i) P^T,\lambda_i \in \mathbb{R}^+$\\
    令$A^{-\frac{1}{2}}=P diag(\frac{1}{\sqrt{\lambda_i}}) P^T,A^{\frac{1}{2}}=P diag(\sqrt{\lambda_i}) P^T;$\\
    $A^{-\frac{1}{2}}$也是实对称正定矩阵,因为其本征值$\frac{1}{\sqrt{\lambda_i}}>0$且$(A^{-\frac{1}{2}})^T=(P diag(\frac{1}{\sqrt{\lambda_i}}) P^T)^T=P diag(\frac{1}{\sqrt{\lambda_i}}) P^T=A^{-\frac{1}{2}}$\\
    同理可知,$A^{\frac{1}{2}}$也是实对称正定矩阵;\\
    又$A^{-\frac{1}{2}} A^{\frac{1}{2}}=1$即$A^{-\frac{1}{2}} = (A^{\frac{1}{2}})^{-1}$\\
    且有$A A^{-\frac{1}{2}}=A^{\frac{1}{2}}$\\
    注意到:$$M=A^{-\frac{1}{2}} A B A^{\frac{1}{2}}=P diag(\frac{1}{\sqrt{\lambda_i}}) P^T P diag(\lambda_i) P^T B P diag(\sqrt{\lambda_i}) P^T=A^{\frac{1}{2}} B A^{\frac{1}{2}}$$\\
    $$\therefore M^T = (A^{\frac{1}{2}} B A^{\frac{1}{2}})^T=(A^{\frac{1}{2}} B A^{\frac{1}{2}})=M$$\\
    考虑二次型,由于$B$是正定的:\\
    $$\bra{x}M\ket{x}=\bra{x}(A^{\frac{1}{2}} B A^{\frac{1}{2}})\ket{x}=\bra{y} B \ket{y} >0$$\\
    其中$\ket{y}=A^{\frac{1}{2}}\ket{x},\bra{y}=\bra{y} {A^{\frac{1}{2}}}^T=\bra{y} A^{\frac{1}{2}}$\\
    因此,$M$也是实对称正定矩阵,其具有全是正实数的本征值\\
    而$M =A^{-\frac{1}{2}} A B A^{\frac{1}{2}} \sim AB$\\
    相似的矩阵具有相同的本征值,从而$AB$具有有全是正实数的本征值.\\
\end{proof}
一个扩展练习如下:
\begin{question}
    若S和T是n阶正定实对称方阵,P为n阶实方阵,试证明:
    $$S-P^TT^{-1}P\text{与}T-PS^{-1}P^{T}$$有相同的正定性,即同为正定或非正定。
\end{question}
\begin{analysis}
    我们看到,这是对S、T对称的结论,但具体的表达式却是不对称的。我们用以下的构造,把S,T放到一个扩大矩阵的对角线上,而把这两个表达式视作分块对角化的结果。事实上,这样的构造虽然说不出道理,但却是常见的,若有余力应当记住这样的构造。
\end{analysis}
\begin{proof}
    注意到:
    $$
    \begin{bmatrix}
        I & -T^{-1}P \\
        0 & I        \\
    \end{bmatrix}^T
    \begin{bmatrix}
        T & P \\
      P^T & S \\
    \end{bmatrix}^T
    \begin{bmatrix}
        I & -T^{-1}P \\
        0 & I        \\
    \end{bmatrix}=
    \begin{bmatrix}
        T & 0            \\
        0 & S-P^TT^{-1}P \\
    \end{bmatrix}
    $$
    $$
    \begin{bmatrix}
        I          & 0 \\
        -S^{-1}P^T & I \\
    \end{bmatrix}^T
    \begin{bmatrix}
        T & P \\
      P^T & S \\
    \end{bmatrix}^T
    \begin{bmatrix}
        I          & 0 \\
        -S^{-1}P^T & I \\
    \end{bmatrix}=
    \begin{bmatrix}
        T-PS^{-1}P^{T} & 0 \\
        0              & S \\
    \end{bmatrix}
    $$证讫。
\end{proof}
为了熟悉这样的构造,有一个这样的练习:
\begin{question}
    若S是n阶半正定实对称方阵,且
    $$S=
    \begin{bmatrix}
    S^1_{r \times r}       & S^2_{r \times (n-r)}     \\
    {S^2_{r \times (n-r)}}^T & S^3_{(n-r) \times (n-r)} \\
    \end{bmatrix}
    $$
    试证明:$$\text{det}[S] \leq \text{det}[S^1] \cdot \text{det}[S^3]$$
\end{question}
提示:先证明若S和T都是半正定是对称方阵,则$\text{det}[S+T] \geq \text{det}[S]$\\
\subsection{习题选讲:几何观点}
\begin{question}
    对任一n阶实对称矩阵A和B,记A的最大本征值为$\Lambda[A]$,最小本征值为$\lambda[A]$。试证:
    $\forall \alpha \in [0,1],$\\ $$\alpha \Lambda[A]+(1-\alpha)\Lambda[B] \geq \Lambda[\alpha A +(1-\alpha) B],$$  $$ \alpha \lambda[A]+(1-\alpha)\lambda[B] \leq \lambda[\alpha A +(1-\alpha) B]$$
\end{question}
\begin{analysis}
    本题直接从代数上求解十分困难,因为其涉及组合矩阵的最大特征值,最大特征值的出现位置难以确定。乍看无从下手,但若我们想起可对角化矩阵的几何意义:伸缩变换;以及特征值的几何意义:主轴伸缩比例;从而得出下面的引理,这个问题就迎刃而解了。
    类似地,SVD的几何意义、行列式的几何意义都应清楚,这有时很有助于解题。
\end{analysis}
\begin{lemma}
    $$\Lambda[A]=\underset{\forall x \in V}{max} [\frac{x^T A x}{\|x\|^2}],\lambda[A]=\underset{\forall x \in V}{min}{\forall x \in V}[\frac{x^T A x}{\|x\|^2}]$$
\end{lemma}
\begin{proof}
    由于A是实对称矩阵,其正交相似于对角阵,故
    $$\frac{x^T A x}{\|x\|^2}=\frac{1}{[x^T P^T][P x]}[x^T P^T] 
    \begin{bmatrix}
        \Lambda   & \null     & \null     & \null     \\
        \null     & \lambda   & \null     & \null     \\
        \null     & \null     & \lambda_3 & \null     \\
        \null     & \null     & \null     & \cdots    \\
        \null     & \null     & \null     & \lambda_n \\
    \end{bmatrix} [P x]$$\\
    令$Px=y=\begin{bmatrix} y_1 \\ y_2 \\ \cdots \\ y_n \end{bmatrix}$\\
    有$$\frac{x^T A x}{\|x\|^2}=\frac{\Lambda y_1^2+\lambda y_2^2+\sum \limits_{i=3}^{n} \lambda_i^2 y_i^2}{\sum \limits_{i=1}^{n} y_i^2} \in [\Lambda,\lambda]$$
\end{proof}
\begin{proof}
    由引理$$\Lambda[\alpha A +(1-\alpha) B]=\underset{\forall x \in V}{max} [\frac{x^T (\alpha A +(1-\alpha) B) x}{\|x\|^2}]=\underset{\forall x \in V}{max} [\alpha [\frac{x^T A x}{\|x\|^2}]+(1-\alpha)[\frac{x^T  B x}{\|x\|^2}]] $$ $$ \leq \alpha \underset{\forall x \in V}{max} [[\frac{x^T A x}{\|x\|^2}]]+(1-\alpha)\underset{\forall x \in V}{max}[[\frac{x^T  B x}{\|x\|^2}]]=\alpha \Lambda[A]+(1-\alpha)\Lambda[B]$$
\end{proof}
\section{其它习题选讲}
    “科学就是物理学和集邮。”——费曼
\begin{question}
(实虚部)\\
设 $\lambda_0$ 是 $n$ 阶实正交方阵 $A$ 的一个特征值(可能是复数)。\\
(1)试证 $|\lambda_0| = 1$\\
(2)假设 $\lambda_0 \notin \mathbb{R}$,$z \in \mathbb{C}^n$ 是 $\lambda_0$ 的一个复特征向量。
记 $z = x + iy$,这里 $x, y \in \mathbb{R}^n$。证明 $x \perp y$ 并且 $\|x\| = \|y\|$。
\end{question}
本题中,$\bra{x}$表示$\ket{x}$的转置共轭;
\begin{proof}
(1)\\
    由矩阵之正交性有,设$\ket{0}$是其本征值为$\lambda_0$的本征矢量,\\
    $\bra{0} A^\dagger A \ket{0}=\braket{0|0}=\braket{0|0} \cdot \lambda_0^* \lambda_0$\\
    $\therefore \lambda_0^* \lambda_0=1$即$|\lambda_0|=1$\\
(2)\\
    设$\lambda_0=\alpha+i\beta$,注意到\\
    $A\ket{z}=A(\ket{x}+i\ket{y})=(\alpha+i\beta)(\ket{x}+i\ket{y})=(\alpha \ket{x}-\beta \ket{y})+i(\beta \ket{x}+\alpha \ket{y})$\\
    由实虚部分别相等有($A$设为实矩阵)\\
    $A\ket{x}=\alpha\ket{x}-\beta\ket{y}$\\
    $A\ket{y}=\alpha\ket{y}+\beta\ket{x}$\\
    同理,由矩阵之正交性有\\
    $\bra{x}A^\dagger=\alpha\bra{x}-\beta\bra{y}$\\
    $\bra{y}A^\dagger=-\alpha\bra{y}-\beta\bra{x}$\\
    $\because \braket{x|y}=\bra{y}A^\dagger A\ket{x}=-\bra{x}A^\dagger A\ket{y}=\bra{x}A^\dagger A\ket{y}$\\
    $\therefore \braket{x|y}=0$,即$\ket{x},\ket{y}$正交.\\
    $\therefore \braket{x|x}=\bra{x} A^\dagger A \ket{x}=(\alpha\bra{x}-\beta\bra{y})(\alpha\ket{x}-\beta\ket{y})=\alpha^2\braket{x|x}+\beta^2\braket{y|y}-2\alpha\beta\braket{x|y}=\alpha^2\braket{x|x}+\beta^2\braket{y|y}$\\
    即$\beta^2\braket{y|y}=(1-\alpha^2)\braket{x|x}$\\
    由(a)中已证$|\lambda_0|=1$\\
    $\therefore 1-\alpha^2=\beta^2$\\
    即$\braket{x|x}=\braket{y|y}$,即$\|x\|=\|y\|$
\end{proof}

\begin{question}
    (矩阵的迹)\\
    设 \( A \) 是 \( n \) 阶反对称方阵,即 \( A^T = -A \)。证明 \( e^A \in SO(n) \) 是 \( n \) 阶特殊正交方阵。
\end{question}
\begin{proof}
    先证正交性:由泰勒展开\\
    $e^A=\sum\limits_{k=0}^{\infty} \frac{A^k}{k!}$\\
    由转置的运算规律及反对称性\\
    $(e^A)^T=\sum\limits_{k=0}^{\infty} \frac{A^{T^k}}{k!}=\sum\limits_{k=0}^{\infty} \frac{(-A)^k}{k!}=e^{A^T}=e^{-A}$\\
    $\therefore e^A\cdot(e^A)^T=1$,即是正交的。\\
    再证行列式为1:\\
    \begin{lemma}
        反对称阵具有纯虚共轭或为零的本征值;
    \end{lemma}
    \begin{proof}
        设$A$具有本征值$\lambda=\alpha+i\beta;\alpha,\beta \in \mathbb{R}$,对应的本征矢量为$\ket{z}=\ket{x}+i\ket{y};\ket{x},\ket{y} \in \mathbb{R}^n$\\
        $\because A\ket{z}=A(\ket{x}+i\ket{y})=(\alpha+i\beta)(\ket{x}+i\ket{y})=(\alpha \ket{x}-\beta \ket{y})+i(\beta \ket{x}+\alpha \ket{y})$\\
        由实虚部分别相等有($A$设实矩阵)\\
        $A\ket{x}=\alpha\ket{x}-\beta\ket{y}$\\
        $A\ket{y}=\alpha\ket{y}+\beta\ket{x}$\\
        $\therefore \bra{x}A\ket{x}=\alpha \braket{x|x}-\beta \braket{x|y}, \bra{y}A\ket{y}=\alpha \braket{y|y}+\beta \braket{x|y}$\\
        两式相加,有\\
        $\bra{x}A\ket{x}+\bra{y}A\ket{y}=\alpha (\braket{x|x}+\braket{y|y})$\\
        同取转置,有:\\
        $(\bra{x}A\ket{x}+\bra{y}A\ket{y})^T=\alpha (\braket{x|x}+\braket{y|y})$\\
        而考虑反对称性:\\
        $(\bra{x}A\ket{x}+\bra{y}A\ket{y})^T=\bra{x} A^T \ket{x}+\bra{y} A^T \ket{y}=-(\bra{x}A\ket{x}+\bra{y}A\ket{y})$\\
        $\therefore \bra{x}A\ket{x}+\bra{y}A\ket{y}=\alpha (\braket{x|x}+\braket{y|y})=0$\\
        由内积之正定性,$\alpha=0$,即是纯虚的.\\
        而考虑$A$是实矩阵,其特征方程是实系数多项式方程,解必为共轭复数或0;\\
        $\therefore$反对称阵具有纯虚共轭或为零的本征值;\\
    \end{proof}
    由引理,及约当标准型,\\
    $\exists P$可逆,$A=P\begin{bmatrix}
        \lambda_1 & *         & \cdots    & *         \\
        \null     & \lambda_2 & \cdots    & *         \\
        \null     & \null     & \lambda_3 & \cdots    \\
        \null     & \null     & \null     & \cdots    \\
        \null     & \null     & \null     & \lambda_n \\
    \end{bmatrix}P^{-1}=P \Lambda P^{-1}$\\
    其中,$\lambda_i$是之本征值,由泰勒展开:\\
    $e^A=\sum\limits_{k=0}^{\infty} \frac{A^k}{k!}=\sum\limits_{k=0}^{\infty} P \frac{\Lambda^k}{k!} P^{-1}$\\
    而$\Lambda^k=\begin{bmatrix}
        \lambda_1^k & *           & \cdots      & *           \\
        \null       & \lambda_2^k & \cdots      & *           \\
        \null       & \null       & \lambda_3^k & \cdots      \\
        \null       & \null       & \null       & \cdots      \\
        \null       & \null       & \null       & \lambda_n^k \\
    \end{bmatrix}$\\
    $\therefore e^A=P\begin{bmatrix}
        e^\lambda_1 & *           & \cdots      & *           \\
        \null       & e^\lambda_2 & \cdots      & *           \\
        \null       & \null       & e^\lambda_3 & \cdots      \\
        \null       & \null       & \null       & \cdots      \\
        \null       & \null       & \null       & e^\lambda_n \\
    \end{bmatrix}P^{-1}$\\
    $\therefore \text{det}(e^A)=\text{det}(P) exp[\sum\limits_{k=1}^{n} \lambda_k] \text{det}(P^{-1})$\\
    由引理,$\sum\limits_{k=1}^{n} \lambda_k=0$,即$\text{det}(e^A)=1$\\
    $\therefore e^A \in SO(3)$
\end{proof}
以上证明过于笨拙,我们可以用迹的性质来简化行列式为1的证明(大家应该都是这么做的,但这是用于复习的好例子)。
\begin{lemma}
    对$A \in F^{n \times n}$,$$\text{det}[e^A]=e^{\text{tr}[A]}$$
\end{lemma}
为了证明这个引理,我们需要引入迹的运算性质。
\begin{lemma}\ \\
    (1)$\text{tr}[A+B]=\text{tr}[A]+\text{tr}[B],\text{tr}[A]^*=\text{tr}[A^{\dagger}]$\\
    (2)(轮换性)$\text{tr}[AB]=\text{tr}[BA],\text{tr}[ABC]=\text{tr}[BCA]=\text{tr}[CAB] \cdots$\\
    (3)(相似不变迹)若$P$可逆,$\text{tr}[A]=\text{tr}[P^{-1}AP]$\\
\end{lemma}
\begin{proof}\ \\
    (1)\\显然;\\
    (2)\\
    $\text{tr}[\prod \limits_{s=1}^{l}A^{(s)}]=\sum \limits_{i_1, i_2,\cdots,i_l} a^{(1)}_{i_1i_2 } a^{(2)}_{i_2i_3} \cdots a^{(l)}_{i_li_1}=\sum \limits_{i_1, i_2,\cdots,i_l} a^{(2)}_{i_2i_3 } a^{(3)}_{i_3i_4} \cdots a^{(l)}_{i_li_1}a^{(1)}_{i_1i_2 }=\text{tr}[[\prod \limits_{s=2}^{l}A^{(s)}]A^{(1)}]$\\
    证讫。\\
    (3)\\
    由(2),$\text{tr}[P^{-1}AP]=\text{tr}[APP^{-1}]=\text{tr}[AI]=\text{tr}[A]$证讫。
\end{proof}
%注意标号的变动以调整此处
对引理6的证明:
\begin{proof}
    由约当标准型,A可以相似地化为上三角形式\\
    $$A=P\begin{bmatrix}
        \lambda_1 & *         & \cdots    & *         \\
        \null     & \lambda_2 & \cdots    & *         \\
        \null     & \null     & \lambda_3 & \cdots    \\
        \null     & \null     & \null     & \cdots    \\
        \null     & \null     & \null     & \lambda_n \\
    \end{bmatrix}P^{-1}=P \Lambda P^{-1}$$\\
    $$\therefore e^A=P\begin{bmatrix}
        e^\lambda_1 & *           & \cdots      & *           \\
        \null       & e^\lambda_2 & \cdots      & *           \\
        \null       & \null       & e^\lambda_3 & \cdots      \\
        \null       & \null       & \null       & \cdots      \\
        \null       & \null       & \null       & e^\lambda_n \\
    \end{bmatrix}P^{-1}$$\\
    $\therefore \text{det}[e^A]=e^{\sum \limits_{i=1}^{n}\lambda_i}=e^{\text{tr}[\Lambda]}=e^{tr[A]}$\\
\end{proof}
从而,对原问题的证明就要简单的多。
\begin{proof}
    由于A是反对称矩阵,$\text{tr}[A]=0$\\
    $\therefore \text{det}[e^A]=e^0=1$\\
\end{proof}

与迹有关的其它习题:(homework14\ 3.\ 7.)
\begin{question}
    设 \( A, B \) 是 \( n \) 阶厄密方阵。证明 \( \text{tr}(AB)^2 \) 和 \( \text{tr}(A^2B^2) \) 都是实数并且\[ \text{tr}(AB)^2 \leq \text{tr}(A^2B^2) \]等号成立当且仅当 \( AB = BA \)。
\end{question}
\begin{proof}
    首先证实数,由于AB是厄密矩阵
    $$\text{tr}[ABAB]^*=\text{tr}[(ABAB)^{\dagger}]=\text{tr}[BABA]=\text{tr}[ABAB] \in \mathbb{R}$$\\
    $$\text{tr}[AABB]^*=\text{tr}[(AABB)^{\dagger}]=\text{tr}[BBAA]=\text{tr}[AABB] \in \mathbb{R}$$\\
    再证不等式:结合我们之前提到的对称化构造思想,这里要研究差值,故采取反称构造$C=iAB-iBA$\\
    $S=C^{\dagger}C=-ABAB-BABA+ABBA+BAAB,\text{tr}[C'^2]=-2\text{tr}[ABAB]+2\text{tr}[AABB]$,只需证明C'正定即可。而如我们所希望的,$S$具有这样的形式:$S=C^{\dagger}C$,正定性是易于证明的。
\end{proof}

\begin{question}
    设 \( n \) 阶复方阵 \( A \) 的所有特征值为 \( \lambda_1, \lambda_2, \ldots, \lambda_n \)。证明\[ \sum_{i=1}^{n} |\lambda_i|^2 \leq \text{tr}(A^{\dagger}A) \]等号成立当且仅当 \( A \) 是正规矩阵。
\end{question}
本题十分简单,只需用到相似不变迹的性质。权当一个小练习:
\begin{proof}
    由讲义命题,A可以酉相似地化为上三角形式
    $$S^{-1}AS=[a_{ij}]=
    \begin{bmatrix}
        \lambda_1 & a_{12}    & \cdots & a_{1n}    \\
        \null     & \lambda_2 & \cdots & a_{2n}    \\
        \null     & \null     & \cdots & \null     \\
        \null     & \null     & \cdots & \lambda_n \\
    \end{bmatrix}
    $$
    $\therefore \text{tr}[A^{\dagger}A]=\text{tr}[S^{-1}A^{\dagger}AS]=\sum_{i,j} a_{ij} a_{ij}^* \geq \sum_{i} a_{ii} a_{ii}^*=\sum \limits_{i=1}^{n} |\lambda_i|^2 $\\
    当且仅当A为正规矩阵时,可以对角化,非对角元全为零,取得等号。\\
\end{proof}
最后一个练习:
\begin{question}
    试证:若A,B为n阶实方阵,则
    $$\text{tr}[AB(AB)^T] \leq \text{tr}[AA^T]\cdot \Lambda_{BB^T} $$
    其中$\Lambda_{BB^T}$是$BB^T$的最大特征根。
\end{question}
\begin{proof}
    轮换到需要的形式即可。
    $$\text{tr}[AB(AB)^T]=\text{tr}[ABB^TA^T]=\text{tr}[(A^TA)(BB^T)]$$
    考虑$BB^T$是正定实方阵,正交对角化,有:
    $$\text{tr}[(A^TA)(BB^T)]=\text{tr}[P(A^TA)P^TP(BB^T)P^T]=\text{tr}[P(A^TA)P^T \Lambda]=\text{tr}[A' \Lambda]=\sum \limits_{i=1}^{n} a'_{ii} \lambda_i \leq \Lambda_{BB^T} \sum \limits_{i=1}^{n} a'_{ii}$$
    而$$\sum \limits_{i=1}^{n} a'_{ii}=\text{tr}[P(A^TA)P^T]=\text{tr}[A^TA]=\text{tr}[AA^T]$$
    $$\therefore \text{tr}[AB(AB)^T] \leq \text{tr}[AA^T]\cdot \Lambda_{BB^T}$$
\end{proof}
\section{早期结论复习}
    现将一些早期而常用的结论罗列如下,便于复习。\\
    1.$\text{det}[I-AB]=\text{det}[I-BA]$\\
    2.$\not \exists \text{方阵}A,B,\text{s.t.} AB-BA = I$\\
    3.存在可逆 \( m \) 阶方阵 \( P \) 和可逆 \( n \) 阶方阵 \( Q \) 使得 $ A = P \begin{bmatrix} I_r & 0 \\ 0 & 0 \end{bmatrix} Q $(相抵标准型)\\
    4.上三角行列式的逆仍是上三角行列式\\
    5.(范德蒙德行列式?)
\section{写在最后}
    笔者是清华大学24级物理系的一名同学,本文依据李思老师:线性代数(理科类)课程上课内容和笔者所思所想结合而来。结合李老师课堂讲义阅读体验更佳。
    笔者的本意是希望通过一些做题经验感触,更有逻辑地叙述线性代数,并结合笔者所熟悉的知识,使得线性代数更加具体而非抽象,这也是标题叫Essential Linear Algebra的原因。
    笔者无意编写一套严谨的数学笔记(毕竟是物理系的),希望看上去更像“欧拉”而非“柯西”,这是有助于笔者这种平凡人类理解的。\\   
    \indent 第二版修改了部分错误,增补了DN分解、结论汇总和部分习题。特别感谢刘宏润、刘乐融、周启轩、王博樊、曹乐宸(复旦)等同学和助教老师的对本文的建议、纠错和帮助。\\
    \indent 限于笔者水平,此笔记必有诸多谬误疏漏有待指正,以及不足之处冀望各位高水平人士为之补全。笔者万分荣幸,感激不尽。
    如果你觉得这篇笔记对你有帮助,且愿意请我和咖啡的话,我个人%和下面的二维码都
    是十分欢迎的。\\
    \ \\ \ \\
    \includegraphics[width=\textwidth]{5.1.png}\\
\end{document}
